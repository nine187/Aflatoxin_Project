I modeled the distribution of aflatoxin production in \textit{A. flavus} in maize to determine the trends of changes in suitability within the agricultural areas using historical and future datasets. There was a significant trend of change in the habitat suitability shown in the historical dataset; by 2100, under SRES A2 climate scenario, areas in northern Europe and eastern Russia would be suitable for \\ \textit{A. flavus} to inhabit, given the parameters used. Comparing this future scenario map with the CLIMEX model of maize suitability areas by Ramirez-Cabral et al., (2017) showed that maize cultivation could be at risk of aflatoxin produced by \texit{A. flavus} if moved in the areas mentioned. Aflatoxin is a potent mycotoxin responsible for many liver cancer cases each year \citep{liu2010global}. A complete understanding of different aspects of the species, be it ecologically or genetically, could be useful in helping us understand and reduce the burden of its impact on health. There is also a high economic benefit from reducing aflatoxin levels as this could increase food security globally \citep{gbashi2018socio}. Not many pathogen models were being parameterized on CLIMEX, and this study models the species \textit{A. flavus} within the context of maize and able to explore its distribution. Despite the limitations, various highly significant variables were considered, making this a reasonably reliable model to build upon in future studies. The distribution of this fungus was highly dependent on climate; thus, by understanding its distribution fully, we could mitigate its effect, reduce the global burden of crop loss, and reduce health impacts from the toxin produced by the species.