\subsection{EI historical data map}
\begin{figure*}[!ht]
	\centering
	\includegraphics[width=16cm]{images/EI_map.pdf}
	\caption{Historical EI map for irrigated scenario}
	\label{fig:Historical EI map for irrigated scenario}
\end{figure*}


\pagebreak
\subsection{CLIMEX parameter sensitivity}
\begin{table}[ht]
\caption{CLIMEX Parameters sensitivity for each parameter}
\begin{tabular}{lllllllllllll}
\hline
Param. & Low   & Default & High  & Run   & Range & EI   & GI   & MI    & TI   & DS   & HS   & CS   \\ \hline
\\
SM0   & 0.00  & 0.10    & 0.20  & 1.00  & 3.93  & 4.57 & 5.18 & 23.18 & 0.00 & 0.00 & 0.00 & 0.00 \\
SM1   & 0.10  & 0.20    & 0.30  & 2.00  & 0.80  & 3.61 & 4.17 & 20.87 & 0.00 & 0.00 & 0.00 & 0.00 \\
SM2   & 0.70  & 0.80    & 0.90  & 3.00  & 0.01  & 1.61 & 1.65 & 4.88  & 0.00 & 0.00 & 0.00 & 0.00 \\
SM3   & 1.90  & 2.00    & 2.10  & 4.00  & 0.01  & 1.83 & 1.84 & 2.90  & 0.00 & 0.00 & 0.00 & 0.00 \\
DV0   & 9.00  & 10.00   & 11.00 & 5.00  & 0.16  & 0.83 & 1.33 & 0.00  & 2.04 & 0.00 & 0.00 & 0.00 \\
DV1   & 29.00 & 30.00   & 31.00 & 6.00  & 0.05  & 1.63 & 1.74 & 0.00  & 3.44 & 0.00 & 0.00 & 0.00 \\
DV2   & 34.00 & 35.00   & 36.00 & 7.00  & 0.04  & 0.90 & 0.91 & 0.00  & 2.52 & 0.00 & 0.00 & 0.00 \\
DV3   & 39.00 & 40.00   & 41.00 & 8.00  & 0.06  & 0.45 & 0.45 & 0.00  & 2.13 & 0.00 & 0.00 & 0.00 \\
TTCS  & 9.00  & 10.00   & 11.00 & 9.00  & 1.21  & 1.13 & 0.00 & 0.00  & 0.00 & 0.00 & 0.00 & 7.23 \\
THCS  & 0.00  & 0.00    & 0.00  & 10.00 & 1.30  & 0.99 & 0.00 & 0.00  & 0.00 & 0.00 & 0.00 & 6.43 \\
TTHS  & 39.00 & 40.00   & 41.00 & 11.00 & 0.00  & 0.04 & 0.00 & 0.00  & 0.00 & 0.00 & 0.23 & 0.00 \\
THHS  & 0.00  & 0.00    & 0.00  & 12.00 & 0.00  & 0.02 & 0.00 & 0.00  & 0.00 & 0.00 & 0.10 & 0.00 \\
SMDS  & 0.00  & 0.10    & 0.20  & 13.00 & 0.07  & 0.21 & 0.00 & 0.00  & 0.00 & 8.99 & 0.00 & 0.00 \\
HDS   & 0.00  & 0.00    & 0.00  & 14.00 & 0.00  & 0.06 & 0.00 & 0.00  & 0.00 & 1.32 & 0.00 & 0.00 \\
SMWS  & 1.90  & 2.00    & 2.10  & 15.00 & 0.00  & 0.09 & 0.00 & 0.00  & 0.00 & 0.00 & 0.00 & 0.00 \\
HWS   & 0.00  & 0.00    & 0.00  & 16.00 & 0.00  & 0.05 & 0.00 & 0.00  & 0.00 & 0.00 & 0.00 & 0.00
\end{tabular}
\hline
\vspace{1cm}
\pagebreak
\label{tab:CLIMEX Parameters Sensitivity}
\end{table}

\subsection{CliMond bioclim variables description}
\begin{table}
\caption{Description of CliMond variables}\citep{hutchinson2009anuclim} \citep{kriticos2012climond} \citep{kriticos2014extending}
\centering
\begin{tabular}{ht}
\\
Variable Number & Variable \\
\hline
Bio01 & Annual mean temperature (°C) \\
Bio02 & Mean diurnal temperature range (°C) \\
Bio03 & Isothermality (Bio02 ÷ Bio07) \\
Bio04 & Temperature seasonality (C of V) \\
Bio05 & Max temperature of warmest week (°C) \\
Bio06 & Min temperature of coldest week (°C) \\
Bio07 & Temperature annual range (Bio05-Bio06) (°C) \\
Bio08 & Mean temperature of wettest quarter (°C) \\
Bio09 & Mean temperature of driest quarter (°C) \\
Bio10 & Mean temperature of warmest quarter (°C) \\
Bio11 & Mean temperature of coldest quarter (°C) \\
Bio12 & Annual precipitation (mm) \\
Bio13 & Precipitation of wettest week (mm) \\
Bio14 & Precipitation of driest week (mm) \\
Bio15 & Precipitation seasonality (C of V) \\
Bio16 & Precipitation of wettest quarter (mm) \\
Bio17 & Precipitation of driest quarter (mm) \\
Bio18 & Precipitation of warmest quarter (mm) \\
Bio19 & Precipitation of coldest quarter (mm) \\
Bio20 & Annual mean radiation (W  m$^2$) \\
Bio21 & Highest weekly radiation (W  m$^2$) \\
Bio22 & Lowest weekly radiation (W  m$^2$) \\
Bio23 & Radiation seasonality (C of V) \\
Bio24 & Radiation of wettest quarter (W  m$^2$) \\
Bio25 & Radiation of driest quarter (W  m$^2$) \\
Bio26 & Radiation of warmest quarter (W  m$^2$) \\
Bio27 & Radiation of coldest quarter (W  m$^2$) \\
Bio28 & Annual mean moisture index \\
Bio29 & Highest weekly moisture index \\
Bio30 & Lowest weekly moisture index \\
Bio31 & Moisture index seasonality (C of V) \\
Bio32 & Mean moisture index of wettest quarter \\
Bio33 & Mean moisture index of driest quarter \\
Bio34 & Mean moisture index of warmest quarter \\
Bio35 & Mean moisture index of coldest quarter \\
Bio36 & First principal component of the first 35 Bioclim variables \\
Bio37 & Second principal component of the first 35 Bioclim variables \\
Bio38 & Third principal component of the first 35 Bioclim variables \\
Bio39 & Fourth principal component of the first 35 Bioclim variables \\
Bio40 & Fifth principal component of the first 35 Bioclim variables \\
\hline
\end{tabular}
\label{tab:variables}
\end{table}
\pagebreak

\subsection{Ensemble SDM variables importance}
\begin{figure*}[!ht]
	\centering
	\includegraphics[width=\textwidth]{images/var_importance.png}
	\caption{Ensemble SDM variables importance}
	\label{fig:Ensemble SDM variables importance}
\end{figure*}

\subsection{Ensemble SDM algorithm heatmap}
\begin{figure*}[!ht]
	\centering
	\includegraphics[width=10cm]{images/ensemble_corr.png}
	\caption{Ensemble SDM correlation heatmap}
	\label{fig:Ensemble SDM correlation heatmap}
\end{figure*}

\pagebreak

\subsection{Ensemble model algorithms evaluation}
\begin{figure*}[!ht]
	\centering
	\includegraphics[width=10cm]{images/mod_eva.png}
	\caption{Ensemble model algorithms evaluation, the color indicates the heatmap intensity}
	\label{fig:Ensemble model algorithms evaluation}
\end{figure*}

\subsection{Trend significance of the composite map in CLIMEX}
\begin{figure*}[!ht]
	\centering
	\includegraphics[width=10cm]{images/trend_sig.png}
	\caption{Trend significance of the composite map in CLIMEX, the right legend explains the habitat suitability level}
	\label{fig:Trend significance of the composite map in CLIMEX}
\end{figure*}

\pagebreak
\subsection{Output of CLIMEX suitabiliy of maize from \citep{ramirez2017global}}
\begin{figure*}[!ht]
	\centering
	\includegraphics[width=\textwidth]{images/ramirez's.png}
	\caption{CLIMEX maize suitability output from \citep{ramirez2017global}}
	\label{fig:CLIMEX maize output}
\end{figure*}


